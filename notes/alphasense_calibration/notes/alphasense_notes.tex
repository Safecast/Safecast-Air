\documentclass[letterpaper]{article}
\usepackage{graphicx}
\usepackage{fullpage}
\usepackage{setspace}
\usepackage[parfill]{parskip}
\vfuzz2pt
\hfuzz2pt
\onehalfspacing

\newcommand{\cgas}{C_{\mathrm{gas}}}
\newcommand{\cgasnorm}{C_{\mathrm{gas,*}}}
\newcommand{\wrk}{w}
\newcommand{\aux}{a}
\newcommand{\wrknorm}{\wrk_{\mathrm{*}}}
\newcommand{\auxnorm}{\aux_{\mathrm{*}}}

\begin{document}

\title{Safecast-Air Notes: Alphasense 4-Electrode Sensors}

\date{}
\maketitle

\section{Calculating gas concentration from the electrode voltages}

In this section I discuss how to calculate the gas concentration, $\cgas$, (in
ppb) from the voltage measured at the working electrode, $\wrk$, and the
voltage measured at the auxiliary electrode, $\aux$.   

I examine the following points:

\begin{itemize}

    \item The form of the factory calibration for the Alphasense 4-electrode
        gas sensors as given in the user manuals for the ISB and 3-way AFE
        signal conditioning boards. 

    \item Under what conditions might we calculate $\cgas \leq 0$? Both Rob and
        I have observed this with current Safecast-Air prototypes and we would like
        to understand why it happens and how to fix it.

    \item Possible methods for re-zeroing a sensor.  

\end{itemize}


For the purposes of these notes I am assuming the working and auxiliary
voltages are being read from a sensor operating off a 5V supply using either
the ISB or 3-way AFE signal conditioning boards. I am also assuming that
the voltages have not been scaled via voltage dividers. Note, the current
Safecast-Air prototype uses a voltage divider to scale the electrode voltages
from the 0-5V range to a 0-1.2V range in order to be sampled by an analog input
on the teensy 3.x using the internal 1.2V reference source. I will consider this
scaling in a later section.  However, for the purposes of this section we will
assume that the measured voltages are in the original 0-5V range. 

Also, note that computing the gas concentration for the $\mathrm{O_3}$ sensor
is special case which requires a paired $\mathrm{NO_2}$ sensor in order to
perform a subtraction. This will also be discussed in a later section.

\subsection{Factory calibration}
The factory calibration, which is used to convert the voltages to gas
concentration,  consists of three constants:
\begin{itemize}
    \item $\wrk_0$ $=$  working electrode total zero (V),
    \item $\aux_0$ $=$  auxiliary electrode total zero (V),
    \item $S$ $=$  the sensor's sensitivity (V/ppb).
\end{itemize}

In the Alphasense documentation (See ISB and 3-way AFE user manuals) the procedure for calculating
$\cgas$ from electrode voltages is given as follows: 

\begin{enumerate}
    \item Zero the working electrode voltage $(\wrk - \wrk_0)$, 
    \item Zero the auxiliary electrode voltage $(\aux - \aux_0)$, 
    \item Correct drift by subtracting the zeroed auxiliary elctrode voltage from the zeroed
        working electrode voltage $(\wrk - \wrk_0) - (\aux - \aux_0)$,  
    
    \item Calculate the gas concentration, $\cgas$,  (in ppb) by dividing by the
        drift corrected voltage by the sensor sensitivity, S. This results in
        the folliwng equation for gas concentration: 

\end{enumerate}

\begin{equation} \label{factorycal}
    %\cgas = (1/S)\left[(\wrk - \wrk_0) - (\aux - \aux_0)\right].
    \cgas = \frac{(\wrk - \wrk_0) - (\aux - \aux_0)}{S}
\end{equation}

\subsection{Linear relationship between $\cgas$ and $\wrk-\aux$}
It is easy to see that equation \ref{factorycal} is really just a simple
linear relationship between the difference in the electrode voltages,
$(\wrk-\aux)$, and the gas concentration, $\cgas$.  As such we can rearrange
this equation into the following form
\begin{equation} \label{linearcal}
    \cgas= \alpha (\wrk - \aux) + \beta
\end{equation}
where the slope, $\alpha$, and offset, $\beta$, are constants given by
\begin{equation}
    \alpha = \frac{1}{S}
\end{equation}
and
\begin{equation}
    \beta = \frac{\aux_0 - \wrk_0}{S}.
\end{equation}

Knowing that we should expect a linear relationship between $\cgas$ and
$(\wrk-\aux)$ should be useful when performing an empirical calibration/test of
the sensor. During such a calibration/test we would most likely measure the
working and auxiliary electrode voltages over a range of different gas
concentrations. After performing such measurements one of the first things we
would probably want to look at is a plot of $\cgas$  vs $(\wrk-\aux)$.  From
the form of the factory calibration we would expect to see a linear
relationship in the plot.  In addition we should be able to perform a  linear
regression on this data in
order to determine empirical values for the calibration coefficients $\alpha$
and $\beta$. 

Notes:
\begin{itemize}

    \item The slope of the linear relationship is given by $1/S $ and thus
        depends only on the sensor's sensitivity.   

    \item The $y$-intercept of the linear relationship, the value of $(\wrk -
        \aux)$ for which we get $\cgas = 0$, is given by $(\wrk_0 - \aux_0)$
        and  depends only on the electrode zeros.  
\end{itemize}

Since the $y$-intercept is only related to the electrode zeros it may be
possible to re-zero the sensor/calibration without using expensive test
equipment - either using clean test gas or knowing the current atmospheric gas
concentration. I will elaborate further on this idea later in this section.


\subsection{Under what conditions would we calculate $\cgas \leq 0$?} 

As Rob and I have both seen negative gas concentrations, $\cgas \leq 0$,
reported by the early Safecast-Air prototypes it is interesting to consider
under what conditions this might happen.  From the $y$-intercept calculated in
the previous section we can see that the $\cgas \leq 0$ whenever

\begin{equation}  \label{zerocond2}
    \wrk-\aux \leq \wrk_0 - \aux_0.
\end{equation}
This can be as read as saying that $\cgas \leq 0$ whenever the difference
between the working and auxiliary voltages is less then the difference between
the working and auxiliary total zeros (in the calibration).  In any case it
is clear that  the values of electronic zeros, $\wrk_0$ and $\aux_0$,  are what
determine at what point this zero crossing occurs. Thus a device which is frequently
reporting negative gas concentration probably has an incorrectly set zero(s). 


Some thoughts:

\begin{enumerate}

    \item I'm not completely sure under what conditions we would expect to see
        this. However, the factory calibration is performed under a certain set
        of conditions, i.e., temperature, humidity, etc.  It is very possible
        that the electronic zeros do depend on external factors to some extent.
        In which case we can expect to see some zeroing issues when the sensor
        is operated outside of the calibration conditions.  It is also possible
        that the working and auxiliary electrode zeros may change over time and
        that the sensors need to be re-zeroed occasionally.

    \item Assuming the linear relationship is valid - what are the consequences
        of an incorrectly set zero?  Because the electrode zeros only effect
        the $y$-intercept and not the slope of the linear relationship an
        incorrectly set zero will result in a constant offset in the calculated
        gas concentration.  If this constant offset is in a negative direction
        then, at sufficiently low (actual) gas concentrations, this will result
        in the sensor erroneously reporting a negative gas concentration.

    \item Since gas concentrations which are less than zero don't make physical
        sense it might be reasonable to consider clamping the $\cgas$ values in
        software so that they are always $\geq 0$. Here I am assuming
        that the cause of the negative gas concentration is due to a problem
        with sensor zero - perhaps due to a change in temperature etc. - and
        not due to some other error.  Since we also report the raw 
        working and auxiliary electrode voltages in the JSON data this won't
        result in unrecoverable data loss. 

\end{enumerate}


\subsection{Re-zeroing the sensor}

Prior to re-zeroing the user manual suggests running the sensor for
at least 6 hours in clean air in order to stabilize.

\subsubsection{Using zero air}
If we have zero air (synthetic or scrubbed/cleaned zero air) available then
re-zeroing the sensor is straight forward and we can follow the procedure
described in the user manual.  

\begin{enumerate}
    \item Apply zero air to the sensor for 20min.  
    \item Record the voltages at both the working and auxiliary electrodes. 
\end{enumerate}

The recorded values are then the new working and auxiliary electrode zeros,
$\wrk_0$ and $\aux_0$.  It looks like ``zero'' air can be purchased in small
bottles for a fairly reasonable price. In order to perform the procedure we
would probably need some sort of gas chamber ... maybe make one?

\subsubsection{Quick and dirty - using a known gas concentration}
Another possibility, which is perhaps a bit more quick and dirty, but might be
used to correct a sensor which has an obvious issue with the sensor zero is as
follows.  Say we know the current  gas concentration $\cgasnorm$ maybe from a
reference sensor or perhaps we are in known ``clean'' air and we can use the
baseline atmospheric concentrations.  After the sensor stabilization period we
record the working and auxiliary electrode voltages  $\wrknorm$ and $\auxnorm$
which result from operating at this known  gas concentration. 
    
We then have 
\begin{equation}
    \cgasnorm = \frac{(\wrknorm - \auxnorm) - (\wrk_0 - \aux_0)}{S}
\end{equation}
We can solve this equation for the $y$-intercept, $(\wrk_0 - \aux_0)$, as follows:
\begin{equation}
    \wrk_0 - \aux_0 = \wrknorm - \auxnorm - S \cgas 
\end{equation}
Since the linear relationship in equation \ref{linearcal}  only depends
on difference $(\wrk_0 - \aux_0)$ and not on the individual values of $\wrk_0$
and $\aux_0$ we can set these values relatively arbitrarily (in our
calibration). As long as the difference is equal to $\wrknorm - \auxnorm - S \cgas$
we get the same linear relationship. Thus a set of ``working'' zeroes might be
defined
by 
\begin{equation}
    \wrk_0 = \frac{ \wrknorm - \auxnorm - S\cgas}{2}
\end{equation}
and 
\begin{equation}
    \aux_0 = - \frac{\wrknorm - \auxnorm - S \cgas}{2}.
\end{equation}


\end{document}
